\documentclass[11pt]{article}
\usepackage{graphicx}
\usepackage[margin=2.5cm]{geometry}
\usepackage{tikz}
\usepackage{indentfirst}
\usepackage{tabularx}
\usepackage{listingsutf8}
\usepackage{color}
\usepackage[portuguese]{babel}

\graphicspath{{./images/}}

\def\checkmark{\tikz\fill[scale=0.4](0,.35) -- (.25,0) -- (1,.7) -- (.25,.15) -- cycle;} 
\setlength{\parskip}{0.5em}

\lstset{
	belowcaptionskip=1\baselineskip,
	captionpos=b,
	frame=tb,
	language=C,
	aboveskip=3mm,
	belowskip=3mm,
	showstringspaces=false,
	columns=flexible,
	basicstyle={\small\ttfamily},
	numbers=none,
	numberstyle=\tiny\color{gray},
	keywordstyle=\color{blue},
	commentstyle=\color{dkgreen},
	stringstyle=\color{mauve},
	breaklines=true,
	breakatwhitespace=true,
	tabsize=3,
	inputencoding=utf8,
	extendedchars=true,
	literate={á}{{\'a}}1 {ã}{{\~a}}1 {à}{{\`a} }1 {Ã}{{\~A}}1 {ó}{{\'o}}1 {Ó}{{\'O}}1 {Í}{{\'I}}1 {í}{{\'i}}1 {é}{{\'e}}1 {ç}{{\c{c}}}1 {Ç}{{\c{C}}}1 {ú}{{\'u}}1
}

\begin{document}
	\begin{titlepage}
	\begin{center}
		
		\includegraphics[width=0.3\textwidth]{logo-isec}
		
		\normalsize
		Licenciatura em Engenharia Informática - Curso Europeu \\
		28 de maio de 2021
		
		\LARGE
		Gestão - 2020/2021
		
		\large
		Doutor Jorge Alexandre Almeida
		
		\vspace{1.5cm}
		
		\includegraphics[width=0.3\textwidth]{logo}

		\vspace{0.5cm}

		\Huge
		\textbf{AquiTerrenos, Lda}
		
		\LARGE
		\textbf{Plano de Negócios}
		
		\vspace*{\fill}
		
		\Large
		
		\begin{tabular}{ccc}
			\textbf{Sofia Janeiro} & \textbf{José Almeida} & \textbf{Rúben Lousada} \\ 
			2019132578 & 2019129077 & 2019126176 \\
		\end{tabular}
	
		\vspace{.5cm}
		
		\vfill
		\vspace*{\fill}
		
		
	\end{center}
\end{titlepage}
	
	\tableofcontents
	\pagebreak
	
	\large
	\section{Sumário Executivo}
	
	\normalsize
	
	Possuir e manter terrenos é algo que faz parte da vida de muitos indivíduos e famílias. Por vezes, por falta de visão geral e/ou de conhecimento acerca dos próprios terrenos que possuem, podem-se gerar desentendimentos e dificuldades na sua localização. Com o crescimento da literacia informática da população, faz todo o sentido a procura da informatização do seu património. Muitas vezes, os proprietários dos terrenos procuram fazê-lo, mas não encontram uma solução viável e simples.
	
	A AquiTerrenos pretende resolver esse problema: tem como finalidade servir como uma pequena base de dados para cada família ou indivíduo, onde pode guardar toda a informação que achar relevante sobre os seus terrenos (como pinhais, eucaliptais, vinhas, etc.), assim como partilhar o acesso à mesma com quem quiser, facilitando o acesso a outros serviços externos.
	
	O nosso objetivo é dar ao cliente uma visão alargada sobre os seus terrenos, que o permita gerir os mesmos da forma mais eficiente possível, assim como partilhar informações dos mesmos à sua família / amigos / cooperativas agrícolas, de modo a fomentar e facilitar a cooperação entre proprietários. Implementaremos funcionalidades como geolocalização, informações topográficas, do solo, da fauna e da flora, visualização 3D e 360ª do terreno, entre outras.
	
	Para tirar o máximo proveito dessas funcionalidades, a AquiTerrenos funcionará também como um mercado de terrenos, que irá usar as informações já inseridas pelo cliente sobre os seus terrenos, caso este queira vender os mesmos. Isto simplifica o processo de venda para utilizadores da funcionalidade descrita nos parágrafos anteriores, pois, num caso ideal, já não será necessária a inserção de informação adicional sobre os terrenos que pretende vender, algo que fará com que o utilizador seja mais propício a usar o mercado da AquiTerrenos, e não qualquer outro.
	
	Esta proposta de solução tem de ser rentável para a empresa. Assim, surge a ideia de criação de uma aplicação, inicialmente grátis, onde possa ser feita essa gestão dos terrenos, assim como interagir com o mercado. Algumas funcionalidades estarão barradas por um pagamento único, outras estarão associadas a subscrições mensais/anuais.
	
	A partir de um pequeno estudo de mercado, observamos que por volta de 25\% das famílias que participaram desse estudo não possuem um registo (em papel ou informatizado) dos terrenos que possuem. Este seria o público-alvo do nosso serviço. Além disso, 90\% reportou interesse em utilizar a nossa aplicação e o mercado associado.
	
	De momento, a empresa encontra-se num estado teórico, sendo o próximo passo o desenvolvimento de um protótipo de modo a estudar ainda mais o mercado e o nosso público alvo. Estimamos que, no fim de um período de 5 anos, teremos um VAL de 1.700.000€, assim como uma TIR de aproximadamente 116\%. Estimamos, também, a recuperação do investimento inicial em 4 anos.
	
	\pagebreak
	
	\large
	\section{Identificação da Empresa}
	
	\normalsize
	
	\begin{center}
		\begin{tabular}{ | l | r | }
			\hline
			Designação Social: & AquiTerrenos, Lda \\
			\hline
			Nº de Contribuinte: & 001122334 \\
			\hline 
			Distrito: & Coimbra \\
			\hline   
			Concelho: & Coimbra \\
			\hline   
			Localidade: & Coimbra \\
			\hline   
			Morada (Sede Social): & Rua Pedro Nunes \\
			\hline   
			Telefone: & 123 456 789 \\
			\hline 
			Fax: & 123 456 789 \\
			\hline 
			URL: & aquiterrenos.com \\
			\hline
			E-mail: & aquiterrenosgeral@gmail.com \\
			\hline 
			Responsável: & José Almeida \\
			\hline 
			Cargo: & CEO \\ 
			\hline
			Móvel: & 123 456 789 \\
			\hline 
			Fax: & 123 456 789 \\
			\hline 
			E-Mail: & a2019129077@isec.pt \\
			\hline 
			Data de Constituição e Inicio da Atividade: & 01-06-2021 \\
			\hline 
			Forma Jurídica: & Sociedade por Quotas \\
			\hline 
			Capital Social: & 3.000€ \\
			\hline
			Principais Accionistas: & José Almeida \\
			& Rúben Lousada  \\
			& Sofia Janeiro  \\
			\hline 
			CAE:  & 58290 \\
			\hline 
		\end{tabular}
	\end{center}

	\pagebreak

	\large
	\section{Anexos}

	\normalsize
	\listoffigures
\end{document}