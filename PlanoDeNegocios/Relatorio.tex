\documentclass[11pt]{article}
\usepackage{graphicx}
\usepackage[margin=2.5cm]{geometry}
\usepackage{tikz}
\usepackage{indentfirst}
\usepackage{tabularx}
\usepackage{listingsutf8}
\usepackage{color}
\usepackage[portuguese]{babel}

\graphicspath{{./images/}}

\def\checkmark{\tikz\fill[scale=0.4](0,.35) -- (.25,0) -- (1,.7) -- (.25,.15) -- cycle;} 
\setlength{\parskip}{0.5em}

\lstset{
	belowcaptionskip=1\baselineskip,
	captionpos=b,
	frame=tb,
	language=C,
	aboveskip=3mm,
	belowskip=3mm,
	showstringspaces=false,
	columns=flexible,
	basicstyle={\small\ttfamily},
	numbers=none,
	numberstyle=\tiny\color{gray},
	keywordstyle=\color{blue},
	commentstyle=\color{dkgreen},
	stringstyle=\color{mauve},
	breaklines=true,
	breakatwhitespace=true,
	tabsize=3,
	inputencoding=utf8,
	extendedchars=true,
	literate={á}{{\'a}}1 {ã}{{\~a}}1 {à}{{\`a} }1 {Ã}{{\~A}}1 {ó}{{\'o}}1 {Ó}{{\'O}}1 {Í}{{\'I}}1 {í}{{\'i}}1 {é}{{\'e}}1 {ç}{{\c{c}}}1 {Ç}{{\c{C}}}1 {ú}{{\'u}}1
}

\begin{document}
	\begin{titlepage}
	\begin{center}
		
		\includegraphics[width=0.3\textwidth]{logo-isec}
		
		\normalsize
		Licenciatura em Engenharia Informática - Curso Europeu \\
		28 de maio de 2021
		
		\LARGE
		Gestão - 2020/2021
		
		\large
		Doutor Jorge Alexandre Almeida
		
		\vspace{1.5cm}
		
		\includegraphics[width=0.3\textwidth]{logo}

		\vspace{0.5cm}

		\Huge
		\textbf{AquiTerrenos, Lda}
		
		\LARGE
		\textbf{Plano de Negócios}
		
		\vspace*{\fill}
		
		\Large
		
		\begin{tabular}{ccc}
			\textbf{Sofia Janeiro} & \textbf{José Almeida} & \textbf{Rúben Lousada} \\ 
			2019132578 & 2019129077 & 2019126176 \\
		\end{tabular}
	
		\vspace{.5cm}
		
		\vfill
		\vspace*{\fill}
		
		
	\end{center}
\end{titlepage}
	
	\tableofcontents
	\pagebreak
	
	\large
	\section{Sumário Executivo}
	
	\normalsize
	
	Possuir e manter terrenos é algo que faz parte da vida de muitos indivíduos e famílias. Por vezes, por falta de visão geral e/ou de conhecimento acerca dos próprios terrenos que possuem, podem-se gerar desentendimentos e dificuldades na sua localização. Com o crescimento da literacia informática da população, faz todo o sentido a procura da informatização do seu património. Muitas vezes, os proprietários dos terrenos procuram fazê-lo, mas não encontram uma solução viável e simples.
	
	A AquiTerrenos pretende resolver esse problema: tem como finalidade servir como uma pequena base de dados para cada família ou indivíduo, onde pode guardar toda a informação que achar relevante sobre os seus terrenos (como pinhais, eucaliptais, vinhas, etc.), assim como partilhar o acesso à mesma com quem quiser, facilitando o acesso a outros serviços externos.
	
	O nosso objetivo é dar ao cliente uma visão alargada sobre os seus terrenos, que o permita gerir os mesmos da forma mais eficiente possível, assim como partilhar informações dos mesmos à sua família / amigos / cooperativas agrícolas, de modo a fomentar e facilitar a cooperação entre proprietários. Implementaremos funcionalidades como geolocalização, informações topográficas, do solo, da fauna e da flora, visualização 3D e 360ª do terreno, entre outras.
	
	Para tirar o máximo proveito dessas funcionalidades, a AquiTerrenos funcionará também como um mercado de terrenos, que irá usar as informações já inseridas pelo cliente sobre os seus terrenos, caso este queira vender os mesmos. Isto simplifica o processo de venda para utilizadores da funcionalidade descrita nos parágrafos anteriores, pois, num caso ideal, já não será necessária a inserção de informação adicional sobre os terrenos que pretende vender, algo que fará com que o utilizador seja mais propício a usar o mercado da AquiTerrenos, e não qualquer outro.
	
	Esta proposta de solução tem de ser rentável para a empresa. Assim, surge a ideia de criação de uma aplicação, inicialmente grátis, onde possa ser feita essa gestão dos terrenos, assim como interagir com o mercado. Algumas funcionalidades estarão barradas por um pagamento único, outras estarão associadas a subscrições mensais/anuais.
	
	A partir de um pequeno estudo de mercado, observamos que por volta de 25\% das famílias que participaram desse estudo não possuem um registo (em papel ou informatizado) dos terrenos que possuem. Este seria o público-alvo do nosso serviço. Além disso, 90\% reportou interesse em utilizar a nossa aplicação e o mercado associado.
	
	De momento, a empresa encontra-se num estado teórico, sendo o próximo passo o desenvolvimento de um protótipo de modo a estudar ainda mais o mercado e o nosso público alvo. Estimamos que, no fim de um período de 5 anos, teremos um VAL de 1.700.000€, assim como uma TIR de aproximadamente 116\%. Estimamos, também, a recuperação do investimento inicial em 4 anos.
	
	\pagebreak
	
	\large
	\section{Apresentação do Negócio}
	\subsection{Identificação da Empresa}
	
	\normalsize
	
	\begin{center}
		\begin{tabular}{ | l | r | }
			\hline
			Designação Social: & AquiTerrenos, Lda \\
			\hline
			Nº de Contribuinte: & 001122334 \\
			\hline 
			Distrito: & Coimbra \\
			\hline   
			Concelho: & Coimbra \\
			\hline   
			Localidade: & Coimbra \\
			\hline   
			Morada (Sede Social): & Rua Pedro Nunes \\
			\hline   
			Telefone: & 123 456 789 \\
			\hline 
			Fax: & 123 456 789 \\
			\hline 
			URL: & aquiterrenos.com \\
			\hline
			E-mail: & aquiterrenosgeral@gmail.com \\
			\hline 
			Responsável: & José Almeida \\
			\hline 
			Cargo: & CEO \\ 
			\hline
			Móvel: & 123 456 789 \\
			\hline 
			Fax: & 123 456 789 \\
			\hline 
			E-Mail: & a2019129077@isec.pt \\
			\hline 
			Data de Constituição e Inicio da Atividade: & 01-06-2021 \\
			\hline 
			Forma Jurídica: & Sociedade por Quotas \\
			\hline 
			Capital Social: & 3.000€ \\
			\hline
			Principais Accionistas: & José Almeida \\
			& Rúben Lousada  \\
			& Sofia Janeiro  \\
			\hline 
			CAE:  & 58290 \\
			\hline 
		\end{tabular}
	\end{center}

	\large
	\subsection{Denominação e Forma Jurídica Adotadas}
	
	\normalsize
	
	A denominação "AquiTerrenos, Lda" permite identificar, de forma distintiva e fácil de memorizar, a empresa. Apesar de não ser um nome adequado à internacionalização, consideramos que a designação escolhida, por estar em português, é ideal para ganhar força no mercado nacional. Posteriormente, terá de ser adaptado a cada país em que a empresa operar.
	
	Pretende-se que a empresa tome a forma jurídica de sociedade por quotas, devido às vantagens provenientes da mesma, nomeadamente a limitada responsabilidade dos sócios ser limitada aos bens afetos à empresa, tendo isso como consequência o baixo risco pessoal dos sócios.
	
	Outra vantagem clara é o baixo capital social necessário à criação da empresa, o que permite o desenvolvimento inicial da solução apresentada sem grande investimento necessário da parte dos sócios.
	
	Inicialmente, a empresa teria 3 sócios, sendo estes também os autores deste documento.
	
	\pagebreak
	
	\large
	\subsection{Historial da Empresa}
	
	\normalsize
	
	Tendo sido identificadas deficiências a nível de gestão e conhecimento das propriedades, a AquiTerrenos surge como possível solução aos mesmos. Tem como objetivo ajudar os clientes na gestão e conhecimento sobre as suas propriedades, tirando, assim, um maior proveito das mesmas. Esperamos crescer a uma escala não só nacional, como também internacional. A empresa será fundada como uma start-up, em Coimbra, por alunos de Engenharia Informática.
	
	A empresa procurará conceber, desenvolver e comercializar uma aplicação para smartphone, que terá, simultaneamente, características de produto e características de serviço. Esta aplicação terá, também, secções destinadas ao comércio online, ou seja, eCommerce.
	
	A nossa equipa é, atualmente, composta por três elementos. Sendo estes estudantes de Informática, a nossa equipa tem as competências necessárias à criação e gestão da solução proposta, por se tratar de uma aplicação. Toda a equipa tem, também, conhecimentos de inglês a níveis fluentes, assim como alguns conhecimentos de alemão e francês.
	
	\vspace{1cm}
	
	\begin{figure}[h]
		\includegraphics[width=0.15\textwidth,keepaspectratio]{jalmeida}
		\label{fig:ja}
		\centering
	\end{figure}
	
	\vspace{0.3cm}
	
	José Almeida nasceu em Coimbra e viveu sempre em Miranda do Corvo. Concluiu os seus estudos secundários no Curso de Ciêcnais e Tecnologia. É agora estudante na Licenciatura de Engenharia Informática - Curso Europeu. Destaca-se pela suas qualidades de liderança e conhecimentos de informática. É fluente em inglês e tem alguns conhecimentos de alemão.
	
	\vspace{1cm}
	
	\begin{figure}[h]
		\includegraphics[width=0.15\textwidth,keepaspectratio]{rlousada}
		\label{fig:rl}
		\centering
	\end{figure}
	
	\vspace{0.3cm}
	
	Nascido em Viseu, Rúben viveu desde pequeno em Aveiro, onde efetuou todo o seu percurso escolar até ao 12º ano. No secundário, frequentou o Curso Profissional de Programação e Gestão de Sistemas Informáticos. Estuda, agora, no Instituto Superior de Engenharia de Coimbra, na Licenciatura de Engenharia Informática - Curso Europeu. Destaca-se ainda mais pelos seus conhecimentos de informática, tendo mais experiência do que o resto da equipa. É fluente em inglês e tem alguns conhecimentos de alemão.
	
	\vspace{1cm}
	
	\begin{figure}[h]
		\includegraphics[width=0.15\textwidth,keepaspectratio]{sjaneiro}
		\label{fig:sj}
		\centering
	\end{figure}
	
	\pagebreak
	
	Sofia viveu, desde sempre, em Coimbra onde concluiu os seus estudos secundários no Curso de Ciências e Tecnologias. Atualmente, estuda no Instituto Superior de Engenharia de Coimbra em Engenharia Informática (Curso Europeu). Destaca-se na equipa devido às suas capacidades de design, sendo importantíssima para a imagem da empresa. É fluente em inglês e tem alguns conhecimentos de francês.
	
	\vspace{1cm}
	
	\large
	\subsection{Visão}
	
	\normalsize
	
	A AquiTerrenos pretende, no futuro, ser líder na área de ajuda à manutenção e gestão de propriedades. Tem como objetivo estar presente em todo o mundo e, assim, facilitar a gestão de terrenos possuídos, quer sejam por pessoas ou empresas a uma escala global.
	
	Acreditamos que uma frase que descreve bem a nossa visão é "Poder sobre o que é seu", pois o objetivo da empresa, a um nível geral, é esse mesmo: dar conhecimento (e, consequentemente, poder) aos proprietários sobre aquilo que possuem.
	
	\large
	\subsection{Missão}
	
	\normalsize
	
	Através de uma aplicação para smartphone, pretendemos dar ao cliente possibilidades de melhorar a qualidade e diminuir o tempo necessário à gestão das suas propriedades.
	
	Através de diversos planos de pagamento na aplicação, pretendemos oferecer aos nossos stakeholders um investimento bastante positivo, quer em termos monetários, no caso dos investidores, quer em termos de tempo despendido, no caso de outros colaboradores.
	
	A empresa pretende oferecer planos de baixo custo, de modo a poder expandir ao máximo o acesso aos nossos produtos e serviços, crescendo o negócio e procurando alcançar, da melhor forma possível, a nossa visão.
	
	\large
	\subsection{Vetores Estratégicos}
	
	\normalsize
	
	
	\large
	\subsection{Localização das Instalações e Descrição do Local}
	
	\normalsize
	
	
	\large
	\subsection{Razões para a escolha da localização}
	
	\normalsize
	
	
	\pagebreak
	
	\large
	\section{Análise do Produto/Serviço}
	
	\normalsize
	
	\large
	\subsection{Descrição Sumária dos Serviços}
	
	\normalsize
	
	
	\large
	\subsection{Vantagens Distintivas}
	
	\normalsize
	
	
	\large
	\subsection{Desenvolvimentos Previsíveis dos Serviços}
	
	\normalsize
	
	
	\large
	\subsection{Tecnologias a Utilizar e Direitos da Propriedade Industrial}
	
	\normalsize
	
	
	\large
	\subsection{Processo Produtivo}
	
	\normalsize
	
	
	\large
	\subsection{Layout das instalações}
	
	\normalsize
	
	
	\pagebreak
	
	\large
	\section{Análise de Mercado}
	
	\normalsize
	
	
	\large
	\subsection{Evolução Histórica e Previsional do Setor (Problemas e Tendências)}
	
	\normalsize
	
	\large
	\subsection{Enquadramento do Negócio no Setor}
	
	\normalsize
	
	
	\large
	\subsection{Caracterização do Mercado Alvo}
	
	\normalsize
	
	
	\large
	\subsection{Análise da Concorrência}
	
	\normalsize
	
	
	\large
	\subsubsection{Identificação}
	
	\normalsize
	
	
	\large
	\subsubsection{Avaliação da Empresa com os seus Principais Concorrentes}
	
	\normalsize
	
	
	\large
	\subsection{Fornecedores}
	
	\normalsize
	
	\pagebreak
	
	\large
	\section{Estratégia de Marketing}
	
	\normalsize
	
	
	\large
	\subsection{Segmentação}
	
	\normalsize
	
	
	\large
	\subsection{Política do Produto/Serviço}
	
	\normalsize
	
	
	\large
	\subsection{O Preço}
	
	\normalsize
	
	
	\large
	\subsection{Distribuição}
	
	\normalsize
	
	
	\large
	\subsection{Promoção}
	
	\normalsize
	
	\pagebreak
	
	\large
	\section{Organização e Gestão}
	
	\normalsize
	
	
	\large
	\subsection{Experiência dos Promotores}
	
	\normalsize
	
	
	\large
	\subsection{Especialização Funcional da Organização}
	
	\normalsize
	
	
	\large
	\subsubsection{Organigrama}
	
	\normalsize
	
	
	\large
	\subsection{Análise da Adequação do Perfil às Funções}
	
	\normalsize
	
	
	\large
	\subsection{Processo de Decisão}
	
	\normalsize
	
	
	\large
	\subsection{Qualificações do Quadro de Recursos Humanos}
	
	\normalsize
	
	
	\large
	\subsection{Gestão de Recursos Humanos}
	
	\normalsize
	
	
	\large
	\subsection{Profissionais Externos}
	
	\normalsize
	
	\pagebreak
	
	\large
	\section{Riscos do Negócio}
	
	\normalsize
	
	
	\large
	\subsection{Análise Externa - Ameaças e Oportunidades}
	
	\normalsize
	
	
	\large
	\subsubsection{Ambiente Geral ou Macroambiente}
	
	\normalsize
	
	
	\large
	\subsubsection{Ambiente da Indústria ou Competitivo}
	
	\normalsize
	
	
	\large
	\subsubsection{Análise Interno - Forças e Fraquezas}
	
	\normalsize
	
	
	\large
	\subsection{Análise SWOT}
	
	\normalsize
	
	
	\large
	\subsection{Modelo das 5 Forças de Porter}
	
	\normalsize
	
	
	\large
	\subsubsection{Ameaça de Novas Entradas}
	
	\normalsize
	
	
	\large
	\subsubsection{Ameaça de Serviços Substitutos}
	
	\normalsize
	
	
	\large
	\subsubsection{Rivalidade Entre os Concorrentes}
	
	\normalsize
	
	
	\large
	\subsubsection{Poder Negocial dos Clientes}
	
	\normalsize
	
	
	\large
	\subsubsection{Poder Negocial dos Fornecedores}
	
	\normalsize
	
	\pagebreak
	
	\large
	\section{Plano de Implementação}
	
	\normalsize
	
	\pagebreak
	
	\large
	\section{Análise da Viabilidade Económica e Financeira}
	
	\normalsize
	
	\large
	\subsection{Pressupostos do Projeto}
	
	\normalsize
	
	
	\large
	\subsection{Investimento e Financiamento Previsionais}
	
	\normalsize
	
	
	\large
	\subsection{Proveitos e Custos Previsionais}
	
	\normalsize
	
	\pagebreak
	
	\large
	\section{Análise de Viabilidade: Cash-Flow, VAL, TIR e PayBack}
	
	\normalsize
	
	\pagebreak
	
	\large
	\section{Análise de Sensibilidade}
	
	\normalsize
	
	
	\pagebreak
	
	\large
	\section{Anexos}
	
	\normalsize
	
	
	\large
	\subsection{Demonstrações Económico-Financeiras}
	
	\normalsize
	
	
	\large
	\subsubsection{Conta Estado e Outros Enter Públicos}
	
	\normalsize
	
	
	\large
	\subsubsection{Demonstrações de Resultados Previsionais}
	
	\normalsize
	
	
	\large
	\subsubsection{Balanços Previsionais}
	
	\normalsize
	
	
	\large
	\subsection{Indicadores}
	
	\normalsize

	\pagebreak

	\listoffigures
\end{document}